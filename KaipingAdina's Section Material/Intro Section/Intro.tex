\documentclass{beamer}
\usetheme{CambridgeUS}
\usepackage{subfigure}
\usepackage{multirow}
% \usepackage{makecell}
\usepackage{graphicx}
\usepackage{xcolor}
\definecolor{liteblue}{rgb}{0.36, 0.54, 0.66}
\newcommand{\lblue}{\textcolor{liteblue}}
\usepackage{tikz}
\usepackage{CJKutf8}
\usepackage{ucs}
\usepackage[utf8]{inputenc}
% \usepackage{co}
\usepackage{multicol}
\usepackage{bm}
\usepackage{array}
\usepackage[normalem]{ulem}
\definecolor{liteblue}{rgb}{0.63,0.71,0.8}
\def\checkmark{\tikz\fill[scale=0.4](0,.35) -- (.25,0) -- (1,.7) -- (.25,.15) -- cycle;} 
\newcommand{\blue}{\textcolor{blue}}
\newcommand{\black}{\textcolor{black}}
\newcolumntype{x}[1]{%
>{\centering\arraybackslash}p{#1}}%
\def\arraybackslash{\let\\\tabularnewline}
\setbeamertemplate{itemize items}[default]
\setbeamertemplate{enumerate items}[default]
\setbeamertemplate{footline}{}
\setbeamertemplate{navigation symbols}{}%remove navigation symbols
%{\hfill\insertframenumber/\inserttotalframenumber}%
%\newcommand{\backupbegin}{
%   \newcounter{framenumberappendix}
%   \setcounter{framenumberappendix}{\value{framenumber}}
%}
%\newcommand{\backupend}{
%   \addtocounter{framenumberappendix}{-\value{framenumber}}
%   \addtocounter{framenumber}{\value{framenumberappendix}} 
%}
\makeatletter
\setbeamertemplate{footline}
{
  \leavevmode%
  \hbox{%
  \begin{beamercolorbox}[wd=.333333\paperwidth,ht=2.25ex,dp=1ex,center]{author in head/foot}%
    \usebeamerfont{author in head/foot}\insertshortauthor~~\beamer@ifempty{\insertshortinstitute}{}{(\insertshortinstitute)}
  \end{beamercolorbox}%
  \begin{beamercolorbox}[wd=.333333\paperwidth,ht=2.25ex,dp=1ex,center]{title in head/foot}%
    \usebeamerfont{title in head/foot}\insertshorttitle
  \end{beamercolorbox}%
  \begin{beamercolorbox}[wd=.333333\paperwidth,ht=2.25ex,dp=1ex,right]{date in head/foot}%
    \usebeamerfont{date in head/foot}\insertshortdate{}\hspace*{2em}
    % \insertframenumber{} / \inserttotalframenumber\hspace*{2ex} % DELETED
  \end{beamercolorbox}}%
  \vskip0pt%
}
\makeatother

\title[COMM 106/206]{Introduction to R}
%
\subtitle{\black{Communication Research Methods}} %Comm 106/206: 
\author[Jennifer Pan]{Jennifer Pan}
%
\institute[Stanford]{Assistant Professor\\
  Department of Communication\\
  Stanford University
  \mbox{ }\\
  \mbox{ }\\
  \mbox{ }\\
  \mbox{ }\\
  \mbox{ }\\
  \mbox{ }\\
  \mbox{ }\\
  \mbox{ }\\
  \mbox{ }\\
  {January 13, 2016}}


\date{}

\begin{document}

\frame{\titlepage}
\date{13 Jan. 2016}

\begin{frame}
 \frametitle<+->{Announcements}
 \begin{itemize}[<+->]
   \item Sections: 
     \begin{itemize}
       \item Thursday 4:30-5:20 pm Bldg 240 Rm 101
       \item Friday 1:30-2:20 pm Bldg 120 Rm 314
       \item Monday 2:30-3:20 pm Bldg 160 Rm 319
     \end{itemize}
   \item Pset 1: on Canvas, due Wed 1/20 before class (email before 3pm to stanfordcommresearchmethods@gmail.com)
 \end{itemize}
\end{frame}

\begin{frame}
 \frametitle<+->{Revised schedule}
 \uncover<2->{
 \includegraphics[width=.9\textwidth]{schedule2.png}}
\end{frame}

\begin{frame}
 \frametitle<+->{Where we are?}
 \uncover<2->{How does research, scientific research, theory, data, concepts, measurement fit together?}
 \pause
 \begin{itemize}[<+->]
   \item Research: pushing the bounds of knowledge about different phenomenon
   \item Theories: explanations about phenomena with properties we like
   \item Concepts: what we call the phenomena being explained and doing the explaining
   \item Data (analysis): needed to test theories, can be quantitative or qualitative
   \item Quantitative data comes in the form of variables, which are measured concepts
 \end{itemize}
 \uncover<8->{To work with quantitative data, we need some sort of tool:}  \uncover<9->{$R$}
\end{frame}

\begin{frame}
 \frametitle<+->{Today: R Intro Part I}
 \begin{itemize}[<+->]
   \item Install R
   \item Install RStudio
   \item Use R as a calculator
   \item Save code (your work) in R Script
   \item Create / manipulate \alert{objects} in R
 \end{itemize}
\end{frame}

\begin{frame}
 \frametitle<+->{Install R}
 \begin{itemize}
   \item cran.r-project.org
   \item google ``cran r"
   \item The Comprehensive R Archive Network
 \end{itemize}
\end{frame}
% demo

\begin{frame}
 \frametitle<+->{Install RStudio}
 \begin{itemize}
   \item www.rstudio.com
   \item RStudio Desktop
   \item Open Source Edition (Free)
   \item Windows: RStudio 0.99.491 - Windows Vista/7/8/10
   \item Mac: RStudio 0.99.491 - Mac OS X 10.6+ (64-bit)
 \end{itemize}
\end{frame}

\begin{frame}
 \frametitle<+->{Intro to R}
 \begin{itemize}
   \item use R as a calculator
   \item save your work in a R Script
   \item create and manipulate objects in R
   \item load dataset into R
 \end{itemize}
\end{frame}


\begin{frame}
 \frametitle<+->{Open RStudio}
 \begin{itemize}
   \item Windows: Start Menu
   \item Mac: Applications
 \end{itemize}
\end{frame}

\begin{frame}
 \frametitle{RStudio}
 \includegraphics[width=.8\textwidth]{rstudio1.png}
\end{frame}

\begin{frame}
 \frametitle<+->{R as calculator}
 \begin{itemize}
   \item Spacing doesn't matter
   \item Order of operations applies
 \end{itemize}
\end{frame}

\begin{frame}
 \frametitle<+->{R Script}
 \begin{itemize}
   \item Do not save the R Console output
   \item Save work as a R Script (File $\rightarrow$ New File $\rightarrow$ R Script)
   \item \# tells R to ignore everything that comes after in that line
   \item Put code and comments in R Script
   \item Send comments from R Script to R Console:
     \begin{itemize}
       \item Click ``Run"
       \item Keyboard shortcut: Ctrl+Enter (Windows)
       \item Keyboard shortcut: Command+Enter (Mac)
     \end{itemize}
   \item Save the R Script
 \end{itemize}
\end{frame}

\begin{frame}
 \frametitle<+->{Objects}
 \begin{itemize}
   \item Shortcuts to some pice of information or data
   \item Save work as a R Script (File $\rightarrow$ New File $\rightarrow$ R Script)
   \item Create objects with assignment operator: {\tt <-}
   \item Use intuitive and informative names:
     \begin{itemize}
       \item Cannot begin with a number, but can contain numbers
       \item Cannot contain spaces
       \item Avoid special characters like \#, \%, \$
       \item Names are case sensitive
     \end{itemize}
   \item Once you have created an object, see in Environment window on the top right (RStudio), see all objects with {\tt ls()}
 \end{itemize}
\end{frame}

\begin{frame}
 \frametitle<+->{Classes}
 \begin{itemize}
   \item R recognizes different types of objects by assigning each object to a class
   \item Allows R to perform appropriate operations on an object depending on its class
     \begin{itemize}
       \item Number is stored as a numeric object
       \item Character string is recognized as a character object
     \end{itemize}
   \item RStudio: Environment will show you the class of objects
   \item Use {\tt class()} to see the class of objects
 \end{itemize}
\end{frame}

\begin{frame}
 \frametitle<+->{Vectors}
 \begin{itemize}
   \item Simplest (not very efficient) way of entering data into R
   \item One-dimensional array representing a collection of information stored in a specific order
   \item Use {\tt c()} to enter a data separated by commas
   \item Indexing: use square brackets {\tt []} to access specific elements of a vector
   \item Arithmetic operations can be done using multiple vectors
 \end{itemize}
\end{frame}

\begin{frame}
 \frametitle<+->{Functions}
 \begin{itemize}
   \item We have seen several functions: sqrt(), class(), c()
   \item Format: {\tt funcname(input)}
     \begin{itemize}
       \item {\tt funcname}: function name
       \item {\tt (input)}: input object (also called arguments)
     \end{itemize}
   \item Useful functions
     \begin{itemize}
       \item {\tt length()}: length of a vector or equivalently the number of its elements
       \item {\tt min()}: min value
       \item {\tt max()}: max value
       \item {\tt range()}: range of data
       \item {\tt mean()}: mean
       \item {\tt sum()}: sum all values in vector
       \item {\tt names()}: access and assign names to elements of a vector
     \end{itemize}
   \item To avoid confusion and problems stemming from the order, specify name of argument
 \end{itemize}
\end{frame}

\begin{frame}
 \frametitle<+->{Your own functions}
 \begin{itemize}
   \item Use {\tt function()} function to create a new function
   \item Spacing does not matter in R
 \end{itemize}
\end{frame}

%\begin{frame}
% \frametitle{Whole pic}
%% \framesubtitle{MA Senate Election (Metaxas and Mustafaraj 2010)}
% \includegraphics[width=.8\textwidth]{crying-baby.jpg}
%\end{frame}
%
%\begin{frame}
% \frametitle<+->{Left pic}
% \begin{minipage}{.48\linewidth}
% \includegraphics[width=\textwidth]{crying-baby.jpg}
% \end{minipage}\hfill
% \begin{minipage}{.48\linewidth}
% \begin{itemize}[<+->]
%   \item TBD
%   \item TBD
% \end{itemize}
% \end{minipage}
%\end{frame}
%
%\begin{frame}
% \frametitle<+->{Right pic}
% \begin{minipage}{.48\linewidth}
% \begin{itemize}[<+->]
%   \item TBD
%   \item TBD
% \end{itemize}
% \end{minipage}\hfill
% \begin{minipage}{.48\linewidth}
% \includegraphics[width=\textwidth]{crying-baby.jpg}
% \end{minipage}
%\end{frame}

%\begin{frame}
% \frametitle<+->{Summing Up}
% \uncover<2->{\alert{Reminders}}
% \pause
% \begin{itemize}[<+->]
%  \item Team Feedback: be specific about your observations
%  \item Lab 3: replicate the \alert{results}
% \end{itemize}
%\end{frame}

\end{document}


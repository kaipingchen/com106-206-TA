\documentclass[handout]{beamer}
\usetheme{CambridgeUS}
\usepackage{subfigure}
\usepackage{multirow}
% \usepackage{makecell}
\usepackage{graphicx}
\usepackage{xcolor}
\definecolor{liteblue}{rgb}{0.36, 0.54, 0.66}
\newcommand{\lblue}{\textcolor{liteblue}}
\usepackage{tikz}
\usepackage{CJKutf8}
\usepackage{ucs}
\usepackage[utf8]{inputenc}
% \usepackage{co}
\usepackage{multicol}
\usepackage{bm}
\usepackage{array}
\usepackage[normalem]{ulem}
\def\checkmark{\tikz\fill[scale=0.4](0,.35) -- (.25,0) -- (1,.7) -- (.25,.15) -- cycle;} 
\newcommand{\blue}{\textcolor{blue}}
\newcommand{\green}{\textcolor{green}}
\newcommand{\black}{\textcolor{black}}
\newcolumntype{x}[1]{%
>{\centering\arraybackslash}p{#1}}%
\def\arraybackslash{\let\\\tabularnewline}
\setbeamertemplate{itemize items}[default]
\setbeamertemplate{enumerate items}[default]
\setbeamertemplate{footline}{}
\setbeamertemplate{navigation symbols}{}%remove navigation symbols
%{\hfill\insertframenumber/\inserttotalframenumber}%
%\newcommand{\backupbegin}{
%   \newcounter{framenumberappendix}
%   \setcounter{framenumberappendix}{\value{framenumber}}
%}
%\newcommand{\backupend}{
%   \addtocounter{framenumberappendix}{-\value{framenumber}}
%   \addtocounter{framenumber}{\value{framenumberappendix}} 
%}
\makeatletter
\setbeamertemplate{footline}
{
  \leavevmode%
  \hbox{%
  \begin{beamercolorbox}[wd=.333333\paperwidth,ht=2.25ex,dp=1ex,center]{author in head/foot}%
    \usebeamerfont{author in head/foot}\insertshortauthor~~\beamer@ifempty{\insertshortinstitute}{}{(\insertshortinstitute)}
  \end{beamercolorbox}%
  \begin{beamercolorbox}[wd=.333333\paperwidth,ht=2.25ex,dp=1ex,center]{title in head/foot}%
    \usebeamerfont{title in head/foot}\insertshorttitle
  \end{beamercolorbox}%
  \begin{beamercolorbox}[wd=.333333\paperwidth,ht=2.25ex,dp=1ex,right]{date in head/foot}%
    \usebeamerfont{date in head/foot}\insertshortdate{}\hspace*{2em}
    % \insertframenumber{} / \inserttotalframenumber\hspace*{2ex} % DELETED
  \end{beamercolorbox}}%
  \vskip0pt%
}
\makeatother

\title[COMM 106/206]{Regression}
%
\subtitle{\black{Communication Research Methods}} %Comm 106/206: 
\author[Jennifer Pan]{Jennifer Pan}
%
\institute[Stanford]{Assistant Professor\\
  Department of Communication\\
  Stanford University
  \mbox{ }\\
  \mbox{ }\\
  \mbox{ }\\
  \mbox{ }\\
  \mbox{ }\\
  \mbox{ }\\
  \mbox{ }\\
  \mbox{ }\\
  \mbox{ }\\
  {February 19, 2016}}


\date{}

\begin{document}

\frame{\titlepage}
\date{19 Feb. 2016}

\begin{frame}
 \frametitle<+->{Announcements}
 \begin{itemize}[<+->]
   \item Monday Feb 29 Section and Tuesday Mar 1 Office Hours
   \item Data for class next Wed
 \end{itemize}
\end{frame}

\begin{frame}
 \frametitle<+->{Where we are}
 \begin{itemize}[<+->]
   \item Previously
     \begin{itemize}
       \item Describing data
       \item Assessing hypotheses by making comparisons
       \item Making comparisons with interval variables (correlation)
     \end{itemize}
   \item Today
     \begin{itemize}
       \item Thinking systematically about how X relates to Y (linear regression)
     \end{itemize}
 \end{itemize}
\end{frame}

\begin{frame}
 \frametitle{16 NBA Players in 2007-008}
 \begin{center}
 \includegraphics[width=.8\textwidth]{blocks1.png}
 \end{center}
\end{frame}

\begin{frame}
 \frametitle{We Might Ask...}
 \begin{center}
 \includegraphics[width=.95\textwidth]{blocks2.png}
 \end{center}
\end{frame}

\begin{frame}
 \frametitle{We Can Make a Scatterplot}
 \begin{center}
 \includegraphics[width=.75\textwidth]{blocks3.png}
 \end{center}
\end{frame}

\begin{frame}
 \frametitle<+->{We Can Calculate Correlation}
 \begin{minipage}{.48\linewidth}
 \includegraphics[width=\textwidth]{blocks3.png}
 \end{minipage}\hfill
 \begin{minipage}{.48\linewidth}
 \begin{itemize}[<+->]
   \item {\tt cor(height, blocks)} = 0.88
   \item Strong, positive (linear) association 
   \item Being tall is associated with blocking more
   \item Being short is associated with blocking less
 \end{itemize}
 \end{minipage}
\end{frame}

\begin{frame}
 \frametitle<+->{We Want to Know More!}
 We might want to ask:
 \begin{enumerate}[<+->]
   \item On average, what is the \blue{effect} of a one inch increase in a player's height on his blocks per game?
   \item If we have a NBA player and know his height, what is his \blue{predicted} blocks per game?
 \end{enumerate}
 \bigskip
 \begin{center}
 \uncover<5->{\textbf{Correlation \alert{cannot} tell us}}
 \bigskip
 \uncover<6->{\textbf{Regression \alert{can} tell us}}
  \end{center}
\end{frame}

\begin{frame}
 \frametitle<+->{Linear Regression}
 \begin{itemize}[<+->]
   \item We will make our \blue{Y} depend on our \green{X} in the following way:
   \uncover<2->{\includegraphics[width=.9\textwidth]{linear1.png}}
   \item This is called a ``linear" function because it produces a straight line graph between \green{X} and \blue{Y}
   \item For any value of \green{X} (and a) we will get an expected value for \blue{Y}
   \item For any value of \green{height} (and a) we will get an expected value for \blue{blocks}
   \item Finding the values for b and a amounts to drawing a `best fit' line for the scatter plot
 \end{itemize}
\end{frame}

\begin{frame}
 \frametitle<+->{Linear Regression: Back to High School}
 \begin{itemize}[<+->]
   \item Suppose $x_1=1$, $y_1=6$ and $x_2=2$, $y_2=8$
   \item What is the relationship between them?
   \uncover<4->{\includegraphics[width=.9\textwidth]{linear2.png}} 
   \pause
   \item \green{y-intercept} = 4, \alert{slope} = 2
 \end{itemize}
\end{frame}

\begin{frame}
 \frametitle{Linear Regression: Back to High School}
 \begin{itemize}[<+->]
   \item Suppose $x_1=1$, $y_1=6$ and $x_2=2$, $y_2=8$
   \item \green{y-intercept} = 4, \alert{slope} = 2
   \bigskip
   \uncover<3->{\includegraphics[width=.9\textwidth]{linear3.png}} 
 \end{itemize}
\end{frame}

\begin{frame}
 \frametitle{Linear Regression: Notation}
   \uncover<2->{\includegraphics[width=\textwidth]{linear4.png}} 
\end{frame}

\begin{frame}
 \frametitle<+->{Back to our NBA Example}
 We want to know:
 \begin{enumerate}
   \item On average, what is the \blue{effect} of a one inch increase in a player's height on his blocks per game?
   \item If we have a NBA player and know his height, what is his \blue{predicted} blocks per game?
 \end{enumerate}
 \uncover<2->{Regression can tell us:} \pause
 \begin{itemize}
   \item the \alert{`best fit'} (straight) line for the data
   \item the \alert{equation} for that line
   \item a predicted Y for any value of X (a predicted blocks per game for any height)
 \end{itemize}
\end{frame}

\begin{frame}
 \frametitle{Back to our NBA Example}
 \begin{center}
 \includegraphics[width=.8\textwidth]{nbaregression.png}
 \end{center}
\end{frame}

\begin{frame}
 \frametitle{Back to our NBA Example}
 \begin{itemize}[<+->]
   \item $\hat{\alpha} = -7.734 $ (the \alert{constant})...
     \begin{itemize}
       \item is the intercept on the y-axis
       \item tells us what value Y takes when X (height) is zero
     \end{itemize}
   \item $\hat{\beta} = 0.108$ (the \alert{coefficient})...
     \begin{itemize}
       \item average change in Y (blocks) for a one unit increase in X (height)
       \item coefficient are always in  terms  of Y's units!
     \end{itemize}
   \item Interpretation of $\hat{\beta}$
      \begin{itemize}
        \item On average...\blue{Y} changes by [\alert{$\hat{\beta}$}][\blue{Y's units}] for a 1 [\green{X's unit}] increase in \green{X}
        \item On average...\blue{a player's blocks per game } changes by [\alert{0.108}][\blue{blocks}] for a 1 [\green{inch}] increase in \green{height}
      \end{itemize}
 \end{itemize}
\end{frame}

\begin{frame}
 \frametitle{Making Predictions}
 \begin{minipage}{.58\linewidth}
 \begin{itemize}[<+->]
   \item To predict \blue{Y} for any \green{X}, just plug the \green{X} into the equation and calculate \blue{$\hat{y}$}
   \item $\hat{y} = \hat{\alpha} +  \hat{\beta} x$
   \item \blue{$\hat{y}$} is the \alert{predicted value}
   \item NBA example:
     \begin{itemize}
       \item $\hat{y} = -7.734 + 0.108 x$
       \item Tony Parker is 6'2" (74in) (not in sample)
       \item $\hat{y} = -7.734 + 0.108 (74) = 0.258$
       \item Parker's \alert{actual} average was 0.1 blocks
     \end{itemize}
 \end{itemize}
 \end{minipage}\hfill
 \begin{minipage}{.38\linewidth}
  \uncover<5->{\includegraphics[width=\textwidth]{tonyparker.jpg}}
 \end{minipage}
\end{frame}

\begin{frame}
 \frametitle<+->{Actual ($y$) vs. Predicted ($\hat{y}$)}
 \begin{minipage}{.52\linewidth}
 \uncover<4->{\includegraphics[width=\textwidth]{y_yhat.png}}
 \end{minipage}\hfill
 \begin{minipage}{.48\linewidth}
 \begin{itemize}[<+->]
   \item $y$ is the actual value of y in the sample
   \item $\hat{y}$ is what we predict based on the regression line
   \item Example: Dwight Howard
     \begin{itemize}
       \item Howard has 2.2 \alert{actual} blocks per game
       \item Howard is \alert{predicted} to have 1.23 from our regression
       \item Difference between actual value and predicted value is the \alert{residual}
       \item Residual for Howard is 2.2 – 1.23 = 0.97
     \end{itemize}
 \end{itemize}
 \end{minipage}
\end{frame}

\begin{frame}
 \frametitle<+->{Calculating the Line}
 \begin{minipage}{.28\linewidth}
 \uncover<3->{\includegraphics[width=.5\textwidth]{nba_residuals.png}}
 \end{minipage}\hfill
 \begin{minipage}{.7\linewidth}
 \begin{itemize}[<+->]
   \item Residual = actual - predicted = $y -\hat{y}$
   \item We have a residual for each observation
   \item We draw the line so it minimizes the residuals
   \item Specifically, minimize the sum of squared residuals
   \item \blue{OLS (ordinary least squares) estimation}
 \end{itemize}
 \end{minipage}
\end{frame}

\begin{frame}
 \frametitle<+->{Linear Regression}
 \begin{itemize}[<+->]
   \item For a given value of X (e.g., height of 67in, 69in)...Y (blocks) can take different values
   \item We will focus on the \alert{mean} value Y takes for a given value of X
   $$\mbox{average}(Y|X) = \alpha + \beta X$$
   \item The slope $\beta$ is always the same...this is a \alert{linear} regression model
   \uncover<5->{\includegraphics[width=.7\textwidth]{linearmodel.png}}
 \end{itemize}
\end{frame}

\begin{frame}
 \frametitle<+->{Ordinary Least Squares}
 \begin{minipage}{.58\linewidth}
 \begin{itemize}[<+->]
   \item In theory, we could use any equation to link y to x: $$\mbox{average}(Y|X) = f(X)$$
   \item Ordinary Least Squares (estimation) is the technique we use to draw the line 
through the points
   \item OLS is so common it's often synonymous with linear regression
   \item But, there are other ways to draw a straight line through points (least absolute values)
 \end{itemize}
 \end{minipage}\hfill
 \begin{minipage}{.38\linewidth}
 \uncover<5->{\includegraphics[width=\textwidth]{olslav.png}}
 \end{minipage}
\end{frame}

\begin{frame}
 \frametitle{Ordinary Least Squares}
 \begin{center}
 \includegraphics[width=.7\textwidth]{ols.png}
 \end{center}
\end{frame}

\begin{frame}
 \frametitle{Linear Regression}
 \begin{itemize}[<+->]
   \item<1-> Units matter (correlation, unit-less) \pause
     \begin{itemize}
       \item The effect of average temperature (X) on ice cream sales (Y)
       \item $\hat{\beta}$ will change if we measure temp in Celcius vs in Fahrenheit (and sales in dollars vs in euros)
       \item \blue{Remember}: $\hat{\alpha}$ and $\hat{\beta}$ are in y's units...not in x's units
     \end{itemize}
   \item<2-> Asymmetric (correlation, symmetric)
     \begin{itemize}
       \item Regression of Y on X is...not the same as regression of X on Y
     \end{itemize}   
 \end{itemize}
\end{frame}

\begin{frame}
 \frametitle{Ordinary Least Squares Linear Regression in R}
 {\tt lm()}
\end{frame}
% Do it in R


%\begin{frame}
% \frametitle{Whole pic}
%% \framesubtitle{MA Senate Election (Metaxas and Mustafaraj 2010)}
% \begin{center}
% \includegraphics[width=.7\textwidth]{graphinvertu.png}
% \end{center}
%\end{frame}
%

%\begin{frame}
% \frametitle<+->{Left pic}
% \begin{minipage}{.48\linewidth}
% \includegraphics[width=\textwidth]{crying.png}
% \end{minipage}\hfill
% \begin{minipage}{.48\linewidth}
% \begin{itemize}[<+->]
%   \item TBD
%   \item TBD
% \end{itemize}
% \end{minipage}
%\end{frame}

%\begin{frame}
% \frametitle<+->{Right pic}
% \begin{minipage}{.48\linewidth}
% \begin{itemize}[<+->]
%   \item TBD
%   \item TBD
% \end{itemize}
% \end{minipage}\hfill
% \begin{minipage}{.48\linewidth}
% \includegraphics[width=\textwidth]{crying.png}
% \end{minipage}
%\end{frame}

%\begin{frame}
% \frametitle<+->{Summing Up}
% \uncover<2->{\alert{Reminders}}
% \pause
% \begin{itemize}[<+->]
%  \item Team Feedback: be specific about your observations
%  \item Lab 3: replicate the \alert{results}
% \end{itemize}
%\end{frame}

\end{document}


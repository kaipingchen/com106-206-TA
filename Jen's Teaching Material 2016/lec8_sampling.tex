\documentclass{beamer}
\usetheme{CambridgeUS}
\usepackage{subfigure}
\usepackage{multirow}
% \usepackage{makecell}
\usepackage{graphicx}
\usepackage{xcolor}
\definecolor{liteblue}{rgb}{0.36, 0.54, 0.66}
\newcommand{\lblue}{\textcolor{liteblue}}
\usepackage{tikz}
\usepackage{CJKutf8}
\usepackage{ucs}
\usepackage[utf8]{inputenc}
% \usepackage{co}
\usepackage{multicol}
\usepackage{bm}
\usepackage{array}
\usepackage[normalem]{ulem}
\def\checkmark{\tikz\fill[scale=0.4](0,.35) -- (.25,0) -- (1,.7) -- (.25,.15) -- cycle;} 
\newcommand{\blue}{\textcolor{blue}}
\newcommand{\black}{\textcolor{black}}
\newcolumntype{x}[1]{%
>{\centering\arraybackslash}p{#1}}%
\def\arraybackslash{\let\\\tabularnewline}
\setbeamertemplate{itemize items}[default]
\setbeamertemplate{enumerate items}[default]
\setbeamertemplate{footline}{}
\setbeamertemplate{navigation symbols}{}%remove navigation symbols
%{\hfill\insertframenumber/\inserttotalframenumber}%
%\newcommand{\backupbegin}{
%   \newcounter{framenumberappendix}
%   \setcounter{framenumberappendix}{\value{framenumber}}
%}
%\newcommand{\backupend}{
%   \addtocounter{framenumberappendix}{-\value{framenumber}}
%   \addtocounter{framenumber}{\value{framenumberappendix}} 
%}
\makeatletter
\setbeamertemplate{footline}
{
  \leavevmode%
  \hbox{%
  \begin{beamercolorbox}[wd=.333333\paperwidth,ht=2.25ex,dp=1ex,center]{author in head/foot}%
    \usebeamerfont{author in head/foot}\insertshortauthor~~\beamer@ifempty{\insertshortinstitute}{}{(\insertshortinstitute)}
  \end{beamercolorbox}%
  \begin{beamercolorbox}[wd=.333333\paperwidth,ht=2.25ex,dp=1ex,center]{title in head/foot}%
    \usebeamerfont{title in head/foot}\insertshorttitle
  \end{beamercolorbox}%
  \begin{beamercolorbox}[wd=.333333\paperwidth,ht=2.25ex,dp=1ex,right]{date in head/foot}%
    \usebeamerfont{date in head/foot}\insertshortdate{}\hspace*{2em}
    % \insertframenumber{} / \inserttotalframenumber\hspace*{2ex} % DELETED
  \end{beamercolorbox}}%
  \vskip0pt%
}
\makeatother

\title[COMM 106/206]{Sampling}
%
\subtitle{\black{Communication Research Methods}} %Comm 106/206: 
\author[Jennifer Pan]{Jennifer Pan}
%
\institute[Stanford]{Assistant Professor\\
  Department of Communication\\
  Stanford University
  \mbox{ }\\
  \mbox{ }\\
  \mbox{ }\\
  \mbox{ }\\
  \mbox{ }\\
  \mbox{ }\\
  \mbox{ }\\
  \mbox{ }\\
  \mbox{ }\\
  {January 29, 2016}}


\date{}

\begin{document}

\frame{\titlepage}
\date{29 Jan. 2016}

\begin{frame}
 \frametitle<+->{Announcements}
 \begin{itemize}[<+->]
   \item No laptop in lecture UNLESS I'm not working in R
   \item Midterm Feb. 5 (traveling - Megan by Feb 3 before class)
   \item Midterm content: today's lecture (Monday Feb. 1 section)
   \item Midterm review Feb 4, Feb 5 section (no section Feb 8)
 \end{itemize}
\end{frame}

\begin{frame}
 \frametitle<+->{Overview}
 \begin{itemize}[<+->]
   \item Up to now
     \begin{itemize}
       \item What is scientific research: theories, concepts, measurements
       \item Types of question we want to answer: causation
       \item Summarizing data: descriptive statistics
       \item Basics of R: tool for working with quantitative data
     \end{itemize}
   \item Today
     \begin{itemize}
       \item Getting representative data: random sampling and pitfalls
     \end{itemize}
 \end{itemize}
\end{frame}

\begin{frame}
 \frametitle<+->{Terminology}
 \begin{minipage}{.48\linewidth}
 \only<1->{\includegraphics[width=\textwidth]{terminology.png}}%
 \end{minipage}\hfill
 \begin{minipage}{.48\linewidth}
 \begin{itemize}[<+->]
   \item Population
   \item Sample
   \item Random Sampling
 \end{itemize}
 \end{minipage}
\end{frame}

\begin{frame}
 \frametitle<+->{Population}
 Population: the \alert{universe of cases} we want to describe
 \begin{itemize}[<+->]
   \item ``What is the average income of adults in the US?"
      \begin{itemize}
        \item \blue{Population}: \underline{every} person in the US, over 18 years of age
      \end{itemize}
   \item ``Does poor diet increase the risk of prostate cancer?"
      \begin{itemize}
        \item \blue{Population}: \underline{all} males (of all ages, in the world)
      \end{itemize}
   \item ``What is the average price increase for goods and services?"
      \begin{itemize}
        \item \blue{Population}: \underline{every} good and service sold, \underline{every} store
      \end{itemize}
   \item ``How will American vote in the 2016 presidential election?
      \begin{itemize}
        \item \blue{Population}: \underline{all} US citizens...who intend to vote? who are eligible to vote?
      \end{itemize}
 \end{itemize}
\end{frame}

\begin{frame}
 \frametitle{Population}
 Population: the \alert{universe of cases} we want to describe
 Population parameter: the characteristic of the population we care about
 \begin{itemize}[<+->]
   \item ``What is the average income of adults in the US?"
      \begin{itemize}
        \item \blue{Population parameter}: average income
      \end{itemize}
   \item ``Does poor diet increase the risk of prostate cancer?"
      \begin{itemize}
        \item \blue{Population parameter}: risk
      \end{itemize}
   \item ``What is the average price increase for goods and services?"
      \begin{itemize}
        \item \blue{Population parameter}: average price
      \end{itemize}
   \item ``How will American vote in the 2016 presidential election?
      \begin{itemize}
        \item \blue{Population parameter}: vote intention
      \end{itemize}
 \end{itemize}
\end{frame}

\begin{frame}
 \frametitle{Population}
 \begin{minipage}{.58\linewidth}
 Population: the \alert{universe of cases} we want to describe
 Population parameter: the characteristic of the population we care about
 \begin{itemize}[<+->]
   \item Census: when we have every single observation from the population
   \item Rare, usually:
      \begin{itemize}
        \item Too expensive (2010 census cost \$13 billion)
        \item Too time consuming
      \end{itemize}
 \end{itemize}
 \end{minipage}\hfill
 \begin{minipage}{.38\linewidth}
 \includegraphics[width=\textwidth]{population.png}
 \end{minipage}
\end{frame}

\begin{frame}
 \frametitle<+->{Sample}
 Population: the \alert{universe of cases} we want to describe
 \begin{itemize}[<+->]
   \item Take a \alert{sample} of n cases from the population
      \begin{itemize}
        \item A sample of US adults
        \item A sample of men
        \item A sample of stores selling goods and services
        \item A sample of eligible voters
      \end{itemize}   
   \item Calculate a \alert{sample statistic}
      \begin{itemize}
        \item Average income of a sample of US adults
        \item Average risk of prostate cancer of the sample of men
        \item Average price of goods and services in the sample of stores
        \item Vote intention of sample of eligible voters
      \end{itemize}
 \end{itemize}
\end{frame}

\begin{frame}
 \frametitle{Sample vs. Population}
 \begin{minipage}{.35\linewidth}
 \begin{itemize}[<+->]
   \item Population: the universe of cases we want to describe
   \item Population parameter: the characteristic of the population we care about
   \item Sample: n cases from the population
   \item Sample statistic: what we use to estimate the population parameter
 \end{itemize}
 \end{minipage}\hfill
 \begin{minipage}{.64\linewidth}
 \begin{itemize}[<+->]
   \item Examples:
      \begin{itemize}
        \item We don't care about how individuals in Gallup poll (sample) are going to vote: do care about what that tells us about US voters in general (population)
        \item Don't care about whether price of a pizza from Avanti (sample) has increased: do care about what this tells us about inflation in general (population
      \end{itemize}
    \item Saying something about the population from the sample is called \alert{statistical inference}
    \item \blue{Recall}: Inference is a part of what makes research scientific: infer something about the world beyond what we observe
 \end{itemize}
 \end{minipage}
\end{frame}

\begin{frame}
 \frametitle{Sample vs. Population}
 \begin{minipage}{.48\linewidth}
 \includegraphics[width=\textwidth]{samplepop.png}
 \pause
 \end{minipage}\hfill
 \begin{minipage}{.48\linewidth}
 \begin{itemize}[<+->]
   \item Unless we have a census, we can never know \alert{for sure} what the population parameter is...
   \item ...but we can \alert{estimate} it (with varying degrees of uncertainty)
   \item \blue{Recall} Capturing uncertainty is a part of what makes research scientific
   \item Afghanistan data: n = 2754 respondents was used to infer experiences of N = 15 million civilians
 \end{itemize}
 \end{minipage}
\end{frame}

\begin{frame}
 \frametitle{How uncertain?}
 \begin{itemize}[<+->]
   \item How uncertain (or accurate) will our estimate of the population parameter be?
   \item Depends on:
     \begin{enumerate}     
      \item How sample was \alert{chosen}: must be \alert{random} sampling
      \item How \alert{large} sample is: \alert{bigger} sample size is better
      \item Characteristics of population parameter itself: how much \alert{variation} the population shows 
     \end{enumerate}
 \end{itemize}
\end{frame}

\begin{frame}
 \frametitle{Random Sampling}
 \begin{itemize}[<+->]
   \item \blue{Simple random sampling}: every member of the population must have an \alert{equal chance} of being chosen: \blue{random selection}  (there are other methods of random sampling)
   \item If there are N units in the population, chance of any particular unit being in the sample must be is:
         $$ \frac{n}{N}$$
     \begin{itemize}     
      \item N = 230 million possible voters, sample of n = 1000: everyone must have $\frac{1,000}{230,000,000}$ chance of being picked for the sample
      \item N = 6999 undergrads at Stanford, sample of n = 100 for a undergrad survey: everyone must have $\frac{100}{6999}$ chance of being picked for the survey sample
     \end{itemize}
   \item Produces sample that is \alert{representative} of the population: if we repeat the same random sampling procedure many times, features of each sample are not identical to those of population, but on average they are identical
   \item Need \alert{sampling frame}: list of individuals in the population to sample from
 \end{itemize}
\end{frame}

\begin{frame}
 \frametitle{Random Sampling}
 \begin{minipage}{.48\linewidth}
 \includegraphics[width=\textwidth]{biasedsurvey.jpg}
 \end{minipage}\hfill
 \begin{minipage}{.48\linewidth}
 \begin{itemize}[<+->]
   \item Random sampling eliminates \alert{biased differences} between population and sample
   \item Get it wrong
      \begin{itemize}
        \item Allow people to self-select into answering
        \item Only include certain types of people in sample
      \end{itemize}
   \item Answers are meaningless!
 \end{itemize}
 \end{minipage}
\end{frame}

\begin{frame}
 \frametitle{Random Sampling Fail}
 \begin{minipage}{.38\linewidth}
 \includegraphics[width=.8\textwidth]{literarydigestlandon.jpg}
 \end{minipage}\hfill
 \begin{minipage}{.58\linewidth}
 Example of getting it wrong:
 \begin{itemize}[<+->]
   \item Literary Digest (LD) poll 1936 to predict Roosevelt vs. Landon election
   \item LD gets names and address from phone records, automobile clubs, own subscribers
   \item LD send out 10 million sample ballots, get 2.4 million back (HUGE)
   \item Prediction:\\ \blue{Roosevelt} 43\%, \alert{Landon} 57\%
   \item Actual:\\  \blue{Roosevelt} 60.8\%, \alert{Landon} 36.5\%
 \end{itemize}
 \end{minipage}
\end{frame}


\begin{frame}
 \frametitle{Random Sampling Fail}
 \begin{minipage}{.3\linewidth}
 What went wrong?\\
 \includegraphics[width=.8\textwidth]{literarydigestlandon.jpg}
 \end{minipage}\hfill
 \begin{minipage}{.7\linewidth}
 \begin{itemize}[<+->]
   \item Were people in the sample reflective of population of voters?
   \item Wrong \alert{sampling frame}: wrong method for defining population LD wanted to study
   \item Assumed it was everyone in phone directory + those in auto clubs + subscribers
   \item If you were poor, rural, semi-literate, what is your probability of being in that sample?
      \begin{itemize}
        \item Was the probability $\frac{n}{N}$ where n = 2.4 million, N = 46 million voters?
        \item NO! It was 0
      \end{itemize}
   \item If you were a rich, city-dwelling lawyer, what is your probability of being in thatsample?
 \end{itemize}
 \end{minipage}
\end{frame}

\begin{frame}
 \frametitle{Random Sampling Fail}
 Another example: Republican Primary 2007
 \includegraphics[width=.8\textwidth]{repprimary2007.jpg}
 \begin{itemize}
   \item CNBC post-debate online poll: ``who won the debate?"
   \item 7,000 respondents: Ron Paul won with 75\% of the vote
 \end{itemize}
\end{frame}

\begin{frame}
 \frametitle{Random Sampling: Afghanistan Data}
 \begin{minipage}{.48\linewidth}
 \includegraphics[width=.8\textwidth]{afghan.png}
 \end{minipage}\hfill
 \begin{minipage}{.48\linewidth}
 \begin{itemize}
   \item Altitude and population in surveyed and non-surveyed villages
   \item Some outliers
   \item Distribution of these two variables is largely similar between the sampled and non-sampled villages
 \end{itemize}
 \end{minipage}
\end{frame}
% GO TO R

\begin{frame}
 \frametitle{Sampling Bias}
 \begin{itemize}
   \item \blue{Sampling bias}: wrong sampling frame (some types of people, car owners, rich, more likely to be included in survey than others)
     \begin{itemize}
       \item How successful is Alcoholics Anonymous? (But who selects into AA? what are their motivations?)
       \item If certain `types' of units are systematically more or less likely to appear in your sample, you probably have a \alert{selection effect}
     \end{itemize}
 \end{itemize}
\end{frame}

\begin{frame}
 \frametitle<+->{Uncertainty}
 \begin{minipage}{.4\linewidth}
 \begin{itemize}[<+->]
   \item Unless we have a census, we can never know \alert{for sure} what the population parameter is...
   \item ...but we can \alert{estimate} it (with varying degrees of uncertainty)
 \end{itemize}
 \end{minipage}\hfill
 \begin{minipage}{.58\linewidth}
 \uncover<4->{Depends on:}
     \begin{enumerate}[<+->]     
      \item How sample was chosen: random sampling to eliminate bias btw sample and population \checkmark
      \item How \alert{large} sample is: \alert{bigger} sample size is better
      \item Characteristics of population parameter itself: how much \alert{variation} the population shows 
     \end{enumerate}
 \end{minipage}
\end{frame}

\begin{frame}
 \frametitle<+->{Random Sampling Error}
 \begin{minipage}{.35\linewidth}
 \begin{itemize}[<+->]
   \item We can avoid bias, but we get some \alert{random sampling error}
     \begin{itemize}
       \item How sample statistic \alert{differs by chance} from population parameter
       \item We know exactly how this error is affecting our estimate (with bias, we never know for sure)
     \end{itemize}
 \end{itemize}
 \end{minipage}\hfill
 \begin{minipage}{.65\linewidth}
 \uncover<5->{Without sampling bias:}\\
 \uncover<6->{Population parameter = sample statistic  + random sampling error}\\
   \begin{itemize}[<+->]     
      \item Random sampling error for average height of students on campus?
        \begin{itemize}
          \item Get random sample of 100 (no bias!)
          \item Measure heights
          \item By chance 38\% of our sample is less than 5'9", but 36\% of true population is less than 5'9"
        \end{itemize}
   \end{itemize}
 \end{minipage}
\end{frame}

\begin{frame}
 \frametitle<+->{Random Sampling Error: Primaries}
 \begin{minipage}{.58\linewidth}
 \includegraphics[width=\textwidth]{reppolls.png}
 \end{minipage}\hfill
 \begin{minipage}{.38\linewidth}
 \begin{itemize}[<+->]
   \item Random sampling error can lead to an overestimate or underestimate relative to true parameter value
   \item Trump 26.8\% is somewhere between 15\% and 40\%
 \end{itemize}
 \end{minipage}
\end{frame}

\begin{frame}
 \frametitle<+->{Random Sampling Error: Population Variation, Sample Size}
 \begin{minipage}{.38\linewidth}
 \begin{itemize}
   \item<2-> More variation in population  $\rightarrow$ more random sampling error
   \item<5-> Bigger sample $\rightarrow$ less random sampling error
 \end{itemize}
 \end{minipage}\hfill
 \begin{minipage}{.58\linewidth}
	\uncover<3->{
	\begin{center}
	\includegraphics[width=\textwidth]{pop1v2.png}	
	\end{center}}
	\only<4>{Which population is likely to be better represented with a sample of 10 people?}
	\only<6>{Population 2 better represented with a sample of 1000 people than 10 people?}
	\only<7>{Benefit to size not linear: doubling the sample size does not make the estimate twice as good}
 \end{minipage}
\end{frame}

\begin{frame}
 \frametitle<+->{Surveys and Sources of bias}
 \begin{minipage}{.58\linewidth}
 \begin{itemize}[<+->]
   \item Sampling bias and random sampling error apply when doing surveys
   \item Other sources of bias in surveys:
     \begin{itemize}
       \item \blue{Unit non-response}: failure to reach selected units (in Afghanistan survey, 2754 out of 3097 sampled respondents agreed to participate $\rightarrow$ 11\% refusal rate)
       \item \blue{Item non-response}:  respondents refuse to answer certain survey questions (in Afghanistan survey, the income variable had a non-response rate of approximately 5\%)
       \item \blue{Mis-reporting}
     \end{itemize}
 \end{itemize}
 \end{minipage}\hfill
 \begin{minipage}{.38\linewidth}
 \includegraphics[width=\textwidth]{surveyprivacy.jpg}
 \end{minipage}
\end{frame}

\begin{frame}
 \frametitle{Surveys and Sources of bias: Misreporting}
 Different types (reasons) for mis-reporting
 \begin{itemize}[<+->]
   \item Social desirability bias: respondents choose an answer that is seen as socially desirable regardless of what they really think
     \begin{itemize}
       \item Asking question about support for foreign forces in Afghanistan sensitive
       \item Will you vote in the next election?
       \item How often do you lie to people close to you?
       \item How many sexual partners have you had?
       \item Would you be upset if a black family moved next door?
     \end{itemize}
   \item Observer effects: people being surveyed are affected (act differently) by being observed
 \end{itemize}
\end{frame}

\begin{frame}
 \frametitle<+->{Summing Up}
 \begin{itemize}[<+->]
  \item We estimate population parameters from a sample
  \item Make sure your sample is random to avoid bias
  \item Random sampling error is a fact of life
  \item 	More variation increases error, larger sample size reduces it
  \item Problems occur with surveys
 \end{itemize}
\end{frame}

%\begin{frame}
% \frametitle{Whole pic}
%% \framesubtitle{MA Senate Election (Metaxas and Mustafaraj 2010)}
% \includegraphics[width=.8\textwidth]{crying-baby.jpg}
%\end{frame}
%
%\begin{frame}
% \frametitle<+->{Left pic}
% \begin{minipage}{.48\linewidth}
% \includegraphics[width=\textwidth]{crying-baby.jpg}
% \end{minipage}\hfill
% \begin{minipage}{.48\linewidth}
% \begin{itemize}[<+->]
%   \item TBD
%   \item TBD
% \end{itemize}
% \end{minipage}
%\end{frame}
%
%\begin{frame}
% \frametitle<+->{Right pic}
% \begin{minipage}{.48\linewidth}
% \begin{itemize}[<+->]
%   \item TBD
%   \item TBD
% \end{itemize}
% \end{minipage}\hfill
% \begin{minipage}{.48\linewidth}
% \includegraphics[width=\textwidth]{crying-baby.jpg}
% \end{minipage}
%\end{frame}

%\begin{frame}
% \frametitle<+->{Summing Up}
% \uncover<2->{\alert{Reminders}}
% \pause
% \begin{itemize}[<+->]
%  \item Team Feedback: be specific about your observations
%  \item Lab 3: replicate the \alert{results}
% \end{itemize}
%\end{frame}

\end{document}


\documentclass{beamer}
\usetheme{CambridgeUS}
\usepackage{subfigure}
\usepackage{multirow}
% \usepackage{makecell}
\usepackage{graphicx}
\usepackage{xcolor}
\definecolor{liteblue}{rgb}{0.36, 0.54, 0.66}
\newcommand{\lblue}{\textcolor{liteblue}}
\usepackage{tikz}
\usepackage{CJKutf8}
\usepackage{ucs}
\usepackage[utf8]{inputenc}
% \usepackage{co}
\usepackage{multicol}
\usepackage{bm}
\usepackage{array}
\usepackage[normalem]{ulem}
\def\checkmark{\tikz\fill[scale=0.4](0,.35) -- (.25,0) -- (1,.7) -- (.25,.15) -- cycle;} 
\newcommand{\blue}{\textcolor{blue}}
\newcommand{\black}{\textcolor{black}}
\newcolumntype{x}[1]{%
>{\centering\arraybackslash}p{#1}}%
\def\arraybackslash{\let\\\tabularnewline}
\setbeamertemplate{itemize items}[default]
\setbeamertemplate{enumerate items}[default]
\setbeamertemplate{footline}{}
\setbeamertemplate{navigation symbols}{}%remove navigation symbols
%{\hfill\insertframenumber/\inserttotalframenumber}%
%\newcommand{\backupbegin}{
%   \newcounter{framenumberappendix}
%   \setcounter{framenumberappendix}{\value{framenumber}}
%}
%\newcommand{\backupend}{
%   \addtocounter{framenumberappendix}{-\value{framenumber}}
%   \addtocounter{framenumber}{\value{framenumberappendix}} 
%}
\makeatletter
\setbeamertemplate{footline}
{
  \leavevmode%
  \hbox{%
  \begin{beamercolorbox}[wd=.333333\paperwidth,ht=2.25ex,dp=1ex,center]{author in head/foot}%
    \usebeamerfont{author in head/foot}\insertshortauthor~~\beamer@ifempty{\insertshortinstitute}{}{(\insertshortinstitute)}
  \end{beamercolorbox}%
  \begin{beamercolorbox}[wd=.333333\paperwidth,ht=2.25ex,dp=1ex,center]{title in head/foot}%
    \usebeamerfont{title in head/foot}\insertshorttitle
  \end{beamercolorbox}%
  \begin{beamercolorbox}[wd=.333333\paperwidth,ht=2.25ex,dp=1ex,right]{date in head/foot}%
    \usebeamerfont{date in head/foot}\insertshortdate{}\hspace*{2em}
    % \insertframenumber{} / \inserttotalframenumber\hspace*{2ex} % DELETED
  \end{beamercolorbox}}%
  \vskip0pt%
}
\makeatother

\title[COMM 106/206]{Assessing Hypotheses}
%
\subtitle{\black{Communication Research Methods}} %Comm 106/206: 
\author[Jennifer Pan]{Jennifer Pan}
%
\institute[Stanford]{Assistant Professor\\
  Department of Communication\\
  Stanford University
  \mbox{ }\\
  \mbox{ }\\
  \mbox{ }\\
  \mbox{ }\\
  \mbox{ }\\
  \mbox{ }\\
  \mbox{ }\\
  \mbox{ }\\
  \mbox{ }\\
  {February 10, 2016}}


\date{}

\begin{document}

\frame{\titlepage}
\date{10 Feb. 2016}

\begin{frame}
 \frametitle<+->{Announcements}
 \begin{itemize}[<+->]
   \item Be sure to set aside time for psets, pset3 out
   \item Midterm, Pset2: did well, Friday return
   \item Getting help: SECTION, office hours (Tuesday 10:30am-12:30pm, Friday 1:00-3:00pm)
   \item Mid Course Survey
 \end{itemize}
\end{frame}

\begin{frame}
 \frametitle<+->{Where we are}
 \begin{itemize}[<+->]
   \item Up to now:
     \begin{itemize}
       \item What is research and its components
       \item Basics of R
       \item Causal inference
       \item Describing data
     \end{itemize}
   \item This week:
     \begin{itemize}
       \item From what to \textit{why?}: assessing \alert{hypotheses}
       \item We must compare: and match method to level of measurement, e.g., cross-tabs, mean comparisons
       \item A picture is worth a thousand words: graphs and relationships
       \item testing our claims:  statistical significance, null hypothesis, measures of association
     \end{itemize}
 \end{itemize}
\end{frame}

\begin{frame}
 \frametitle<+->{Why...}
 \begin{minipage}{.48\linewidth}
 \includegraphics[width=\textwidth]{why.png}
 \end{minipage}\hfill
 \begin{minipage}{.48\linewidth}
 \begin{itemize}[<+->]
   \item ...do some people vote Democrat, while others vote Republican?
   \item ...do women earn less than men?
   \item ...do wars happen?
 \end{itemize}
 \end{minipage}
\end{frame}

\begin{frame}
 \frametitle{Notice}
 \begin{minipage}{.38\linewidth}
 \begin{itemize}
   \item ...do some people vote Democrat, while others vote Republican?
   \item ...do women earn less than men?
   \item ...do wars happen?
 \end{itemize}
 \end{minipage}\hfill
 \begin{minipage}{.58\linewidth}
 \begin{itemize}
   \item we make an explicit observation about a characteristic that varies: vote choice, salary, war...this will become our dependent variable (Y)
   \item we need a causal explanation as an answer which must...
      \begin{enumerate}
        \item describe a connection between the dependent variable and an independent variable (X)
        \item assert a direction for the difference
        \item be testable!
      \end{enumerate}
 \end{itemize}
 \end{minipage}
\end{frame}

\begin{frame}
 \frametitle<+->{Causal Explanations: Gender and Earnings}
 \begin{minipage}{.48\linewidth}
 \begin{itemize}[<+->]
   \item Why do women \alert{earn less} than men, even in the same jobs? (professor v professor, doctor v doctor etc.)
   \item ...because of \alert{sexism}
   \item too vague!
   \item Think about the causal process
 \end{itemize}
 \end{minipage}\hfill
 \begin{minipage}{.48\linewidth}
 \includegraphics[width=\textwidth]{earn_money_hand.jpg}
 \end{minipage}
\end{frame}

\begin{frame}
 \frametitle{Causal Explanation 1}
 \includegraphics[width=\textwidth]{story1.png}
\end{frame}

\begin{frame}
 \frametitle{Causal Explanation 2}
 \includegraphics[width=\textwidth]{story2.png}
\end{frame}


\begin{frame}
 \frametitle{Causal Explanations: Gender and Earnings}
 \begin{minipage}{.58\linewidth}
 \begin{itemize}[<+->]
   \item Very different explanations, but both might be `sexism'
   \item We don't know if the links in the chain (\alert{intervening variables}) are correct, but we could test them (surveys, experiments etc):
     \begin{itemize}
		\item do mothers do more child care than fathers?
		\item does time off lead to a lack of promotion?
		\item do people have notions of appropriate gender behavior? 
		\item do women negotiate with less aggression?
     \end{itemize}
%   \item Being female is not the independent variable!
 \end{itemize}
 \end{minipage}\hfill
 \begin{minipage}{.38\linewidth}
 \includegraphics[width=\textwidth]{earn_money_hand.jpg}
 \end{minipage}
\end{frame}

\begin{frame}
 \frametitle<+->{Causal Explanations: Gender and Earnings:\\Child-care responsibilities}
 \begin{minipage}{.58\linewidth}
 \begin{itemize}[<+->]
   \item Want to know if child-care responsibilities are the cause of the difference.
   \item We would: 
     \begin{itemize}
       \item<4-> separate units on the \alert{independent variable} \uncover<5->{[child-care responsibilities]...}
       \item<6-> ...and compare values on the \alert{dependent variable} \uncover<7->{[salary]...}
     \end{itemize}
   \item If child-care responsibility are the cause of the difference, we should find a \alert{difference}
   \item \blue{Now we have a hypothesis}
 \end{itemize}
 \end{minipage}\hfill
 \begin{minipage}{.38\linewidth}
 \includegraphics[width=\textwidth]{earn_money_hand.jpg}
 \end{minipage}
\end{frame}

\begin{frame}
 \frametitle<+->{Causal Explanations: Gender and Earnings:\\Child-care responsibility and earnings hypothesis}
 A hypothesis is a \alert{testable statement about the empirical relationship between an independent variable and a dependent variable}
 \bigskip
 \uncover<2->{\begin{block}
 In a comparison of \blue{[units of analysis]} those having \blue{[one value on the independent variable]} will be more likely to have \blue{[one value on the dependent variable]} than will those having \blue{[a different value on the independent variable]}.
 \end{block}}
 \bigskip
 \uncover<3->{Get together in a group with people near you: write down the hypotheses for child-care responsibility and earnings (5min - then come up to board)}
\end{frame}

\begin{frame}
 \frametitle<+->{Examples Hypotheses}
 \uncover<2->{\begin{block}
 In a comparison of \blue{[workers]} those having \blue{[fewer child-care responsibilities]} will be more likely to have \blue{[higher salaries]} than will those having \blue{[more child care responsibilities]}.
 \end{block}}
 \bigskip
 \uncover<3->{\begin{block}
 In a comparison of \blue{[nations]} those having \blue{[a less democratic polity]} will be more likely to have \blue{[wars]} than will those having \blue{[a more democratic polity]}.
 \end{block}}
\end{frame}

\begin{frame}
 \frametitle<+->{Hypotheses Recap}
 A hypothesis is a \alert{testable statement about the empirical relationship between an independent variable and a dependent variable}
 \begin{itemize}[<+->]
   \item we state a \alert{relationship} between two variables: dependent and independent
   \item we talk about \alert{comparing} this type of unit with that type
   \item we are \alert{specific} about what we are comparing and on what grounds
   \item we assert \alert{direction}: more of this leads to more of that 
 \end{itemize}
\end{frame}

\begin{frame}
 \frametitle<+->{How to Make Comparisons?}
 \begin{itemize}[<+->]
   \item We choose different methods, depending on the \alert{type of measure} we have
   \item Suppose the \alert{independent variable} is nominal or ordinal, then:
     \begin{itemize}
       \item \blue{Nominal} \alert{dependent variable}: cross-tabulation
       \item \blue{Ordinal} \alert{dependent variable}: cross-tabulation
       \item \blue{Interval} \alert{dependent variable}: comparison of means
     \end{itemize}
 \end{itemize}
\end{frame}

\begin{frame}
 \frametitle<+->{Our hypothesis}
 \begin{minipage}{.48\linewidth}
 \includegraphics[width=\textwidth]{menshave.png}
 \end{minipage}\hfill
 \begin{minipage}{.48\linewidth}
 \begin{itemize}[<+->]
   \item Men shave their face more often than women
   \item Independent variable: \uncover<4->{Gender, } \uncover<5->{nominal}  
   \item<6-> Dependent variable: \uncover<7->{Frequency of shaving face, } \uncover<8->{ordinal}  
   \item Method of comparison: cross-tabulation (cross-tabs)
 \end{itemize}
 \end{minipage}
\end{frame}

\begin{frame}
 \frametitle<+->{Cross Tabulation}
 \begin{minipage}{.48\linewidth}
 Cross tabs show the distribution of cases \alert{across the values of a dependent variable} for cases that have \textit{different values} on the \alert{independent variable}.
 \end{minipage}\hfill
 \begin{minipage}{.48\linewidth}
 \begin{itemize}[<+->]
   \item The independent variable as the columns
   \item The dependent variable as the rows
   \item Calculate percentages of categories of the independent variable (we ask for ``column percentages" [columns sum to 100\%])
   \item Compare percentages across columns for the same value of the dependent variable
 \end{itemize}
 \end{minipage}
\end{frame}

\begin{frame}
 \frametitle{Our hypothesis}
 \begin{minipage}{.48\linewidth}
 \includegraphics[width=\textwidth]{menshave.png}
 \end{minipage}\hfill
 \begin{minipage}{.48\linewidth}
 \begin{itemize}
   \item Men shave their face more often than women
   \item Independent variable: Gender (nominal)
   \item Dependent variable: Frequency of shaving face (ordinal)
   \item Method of comparison: cross-tabulation (cross-tabs)
 \end{itemize}
 \end{minipage}
\end{frame}

\begin{frame}
 \frametitle<+->{Cross Tabulation: Shaving and Gender}
 \includegraphics[width=.8\textwidth]{shavegender1.png}
\end{frame}

\begin{frame}
 \frametitle<+->{Cross Tabulation: Shaving and Gender}
 \includegraphics[width=.8\textwidth]{shavegender2.png}
\end{frame}

\begin{frame}
 \frametitle<+->{Cross Tabulation in R}
 \begin{itemize}[<+->]
   \item Data: shave.csv
   \item Cross tab function: table()
   \item Recode ``shave" variable: ifelse()   
   \item Column percentages: prop.table()
 \end{itemize}
 \bigskip
 \uncover<6->{Men shave their face more often than women: 29 women never shave, but only 1 man never shaves (100\% of women never shave, but only 3.8\% of men never shave)}
\end{frame}

\begin{frame}
 \frametitle{How to Make Comparisons?}
 \begin{itemize}
   \item We choose different methods, depending on the \alert{type of measure} we have
   \item Suppose the \alert{independent variable} is nominal or ordinal, then:
     \begin{itemize}
       \item \blue{Nominal} \alert{dependent variable}: cross-tabulation
       \item \blue{Ordinal} \alert{dependent variable}: cross-tabulation
       \item \blue{Interval} \alert{dependent variable}: comparison of means
     \end{itemize}
 \end{itemize}
\end{frame}

\begin{frame}
 \frametitle<+->{Comparison of Means}
 \begin{minipage}{.48\linewidth}
 \begin{itemize}[<+->]
   \item Use when we have an interval-level dependent variable
   \item Examples:
     \begin{itemize}
       \item income of blacks vs. whites
       \item height of boys vs. girls
       \item feelings towards Trump for voters in different states 
     \end{itemize}
 \end{itemize}
 \end{minipage}\hfill
 \begin{minipage}{.48\linewidth}
 \includegraphics[width=\textwidth]{measureheight.jpg}
 \end{minipage}
\end{frame}

\begin{frame}
 \frametitle{Comparison of Means: Examples}
 \includegraphics[width=.8\textwidth]{avginced.png}
\end{frame}

\begin{frame}
 \frametitle{Comparison of Means: Examples}
 \includegraphics[width=.8\textwidth]{milkconsump.png}
\end{frame}

\begin{frame}
 \frametitle{Making Comparisons by Graphing}
 \begin{center}
 \includegraphics[width=.8\textwidth]{graphcomp.png} 
 \end{center}
\end{frame}


\begin{frame}
 \frametitle{Making Comparisons by Graphing: Positive Relationship}
 \begin{center}
 \includegraphics[width=.75\textwidth]{graphpos.png}
 \end{center}
\end{frame}

\begin{frame}
 \frametitle{Making Comparisons by Graphing: Negative Relationship}
 \begin{center}
 \includegraphics[width=.75\textwidth]{graphneg.png}
 \end{center}
\end{frame}

\begin{frame}
 \frametitle{Making Comparisons by Graphing: Linear Relationships}
 \begin{center}
 \includegraphics[width=.85\textwidth]{graphlinear.png}
 \end{center}
\end{frame}

\begin{frame}
 \frametitle{Making Comparisons by Graphing: Non-linear Relationships}
 \begin{center}
 \includegraphics[width=.8\textwidth]{graphnonlinear.png}
 \end{center}
\end{frame}

\begin{frame}
 \frametitle{Making Comparisons by Graphing: Non-linear Relationships}
 \begin{center}
 \includegraphics[width=.8\textwidth]{graphcurvelinear.png}
 \end{center}
\end{frame}

\begin{frame}
 \frametitle{Making Comparisons by Graphing: Non-linear Relationships}
 \begin{center}
 \includegraphics[width=.7\textwidth]{graphinvertu.png}
 \end{center}
\end{frame}

\begin{frame}
 \frametitle{Making Comparisons by Graphing: Non-linear Relationships}
 \begin{center}
 \includegraphics[width=.85\textwidth]{graphnonlinear2.png}
 \end{center}
\end{frame}

\begin{frame}
 \frametitle{Making Comparisons by Graphing: R}
 plot()
\end{frame}

\begin{frame}
 \frametitle<+->{Summing Up}
 \begin{itemize}[<+->]
       \item Answering questions with \alert{hypotheses}
       \item Making comparison to answer questions, and matching method to type of variable, e.g., cross-tabs, mean comparisons
       \item Making comparisons by graphing
       \item Next time: testing our claims
 \end{itemize}
\end{frame}

%\begin{frame}
% \frametitle{Whole pic}
%% \framesubtitle{MA Senate Election (Metaxas and Mustafaraj 2010)}
% \includegraphics[width=.8\textwidth]{crying-baby.jpg}
%\end{frame}
%


%
%\begin{frame}
% \frametitle<+->{Right pic}
% \begin{minipage}{.48\linewidth}
% \begin{itemize}[<+->]
%   \item TBD
%   \item TBD
% \end{itemize}
% \end{minipage}\hfill
% \begin{minipage}{.48\linewidth}
% \includegraphics[width=\textwidth]{crying-baby.jpg}
% \end{minipage}
%\end{frame}

%\begin{frame}
% \frametitle<+->{Summing Up}
% \uncover<2->{\alert{Reminders}}
% \pause
% \begin{itemize}[<+->]
%  \item Team Feedback: be specific about your observations
%  \item Lab 3: replicate the \alert{results}
% \end{itemize}
%\end{frame}

\end{document}


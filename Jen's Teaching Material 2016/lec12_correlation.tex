\documentclass{beamer}
\usetheme{CambridgeUS}
\usepackage{subfigure}
\usepackage{multirow}
% \usepackage{makecell}
\usepackage{graphicx}
\usepackage{xcolor}
\definecolor{liteblue}{rgb}{0.36, 0.54, 0.66}
\newcommand{\lblue}{\textcolor{liteblue}}
\usepackage{tikz}
\usepackage{CJKutf8}
\usepackage{ucs}
\usepackage[utf8]{inputenc}
% \usepackage{co}
\usepackage{multicol}
\usepackage{bm}
\usepackage{array}
\usepackage[normalem]{ulem}
\def\checkmark{\tikz\fill[scale=0.4](0,.35) -- (.25,0) -- (1,.7) -- (.25,.15) -- cycle;} 
\newcommand{\blue}{\textcolor{blue}}
\newcommand{\black}{\textcolor{black}}
\newcolumntype{x}[1]{%
>{\centering\arraybackslash}p{#1}}%
\def\arraybackslash{\let\\\tabularnewline}
\setbeamertemplate{itemize items}[default]
\setbeamertemplate{enumerate items}[default]
\setbeamertemplate{footline}{}
\setbeamertemplate{navigation symbols}{}%remove navigation symbols
%{\hfill\insertframenumber/\inserttotalframenumber}%
%\newcommand{\backupbegin}{
%   \newcounter{framenumberappendix}
%   \setcounter{framenumberappendix}{\value{framenumber}}
%}
%\newcommand{\backupend}{
%   \addtocounter{framenumberappendix}{-\value{framenumber}}
%   \addtocounter{framenumber}{\value{framenumberappendix}} 
%}
\makeatletter
\setbeamertemplate{footline}
{
  \leavevmode%
  \hbox{%
  \begin{beamercolorbox}[wd=.333333\paperwidth,ht=2.25ex,dp=1ex,center]{author in head/foot}%
    \usebeamerfont{author in head/foot}\insertshortauthor~~\beamer@ifempty{\insertshortinstitute}{}{(\insertshortinstitute)}
  \end{beamercolorbox}%
  \begin{beamercolorbox}[wd=.333333\paperwidth,ht=2.25ex,dp=1ex,center]{title in head/foot}%
    \usebeamerfont{title in head/foot}\insertshorttitle
  \end{beamercolorbox}%
  \begin{beamercolorbox}[wd=.333333\paperwidth,ht=2.25ex,dp=1ex,right]{date in head/foot}%
    \usebeamerfont{date in head/foot}\insertshortdate{}\hspace*{2em}
    % \insertframenumber{} / \inserttotalframenumber\hspace*{2ex} % DELETED
  \end{beamercolorbox}}%
  \vskip0pt%
}
\makeatother

\title[COMM 106/206]{Correlation}
%
\subtitle{\black{Communication Research Methods}} %Comm 106/206: 
\author[Jennifer Pan]{Jennifer Pan}
%
\institute[Stanford]{Assistant Professor\\
  Department of Communication\\
  Stanford University
  \mbox{ }\\
  \mbox{ }\\
  \mbox{ }\\
  \mbox{ }\\
  \mbox{ }\\
  \mbox{ }\\
  \mbox{ }\\
  \mbox{ }\\
  \mbox{ }\\
  {February 17, 2016}}


\date{}

\begin{document}

\frame{\titlepage}
\date{17 Feb. 2016}

\begin{frame}
 \frametitle<+->{Announcements}
 \begin{itemize}[<+->]
   \item Monday Feb 29 Section and Tuesday Mar 1 Office Hours
   \item Midcourse Survey
   \item Grade Distribution
 \end{itemize}
\end{frame}

\begin{frame}
 \frametitle<+->{Midterm Feedback}
 \begin{itemize}[<+->]
   \item 20 responses out of a class of 44 (selection bias?)
   \item Lecture clear and engaging (8 people)
   \item Sections very helpful (4 people)
   \item Go slower when going through R in class (3 people)
   \item Pset expectations not clear (5 people)\\ \uncover<7->{\blue{Median grade is A- on Pset 1 and 2}} \pause
   \item Respondents divided on group activities:
     \begin{itemize}
       \item Group work is what's going well about class and there should be more (3 people)
       \item Group work should be eliminated (2 people)
     \end{itemize}
 \end{itemize}
\end{frame}

\begin{frame}
 \frametitle<+->{Grade Distribution}
 \begin{minipage}{.35\linewidth}
 \begin{itemize}
   \item A+: $>4$
   \item A: 4
   \item A-: $>=3.5$ to $<4$
   \item B+: $>3$ to $<3.5$
   \item B: 3
   \item B-: $>=2.5$ to $<3$
   \item C+: $>2$ to $<2.5$
   \item C: 2
   \item C-: $>=1.5$ to $<2$
 \end{itemize}
 \end{minipage}\hfill
 \begin{minipage}{.6\linewidth}
 \uncover<2->{\includegraphics[width=.95\textwidth]{grading.png}}
 \end{minipage}
\end{frame}

\begin{frame}
 \frametitle<+->{Where we are}
 \begin{itemize}[<+->]
   \item Previously
     \begin{itemize}
       \item Describing data
       \item Assessing hypotheses by making comparisons
     \end{itemize}
   \item This week
     \begin{itemize}
       \item Making comparisons with interval variables (correlation)
       \item Thinking systematically about how X relates to Y (linear regression)
     \end{itemize}
 \end{itemize}
\end{frame}

\begin{frame}
 \frametitle<+->{Making Comparisons}
\begin{table}[!htp]
\centering
\begin{tabular}{|l|l|l|}
\hline
X & Y & How to Compare \\
\hline
\uncover<2->{nominal} & \uncover<2->{nominal} & \uncover<3->{cross-tabulation }\\
\hline
\uncover<4->{nominal}& \uncover<4->{ordinal}& \uncover<5->{cross-tabulation } \\
\hline
\uncover<6->{nominal} & \uncover<6->{interval} & \uncover<7->{comparison of means} \\
\hline
\uncover<8->{ordinal} & \uncover<8->{nominal}& \uncover<9->{cross-tabulation }\\
\hline
\uncover<10->{ordinal}& \uncover<10->{ordinal} & \uncover<11->{cross-tabulation }\\
\hline
\uncover<12->{ordinal} & \uncover<12->{interval} & \uncover<13->{comparison of means} \\
\hline
\uncover<14->{interval} & \uncover<14->{interval} & \only<15>{{\tt plot()}}\only<16>{\alert{correlation}} \\
\hline
\end{tabular}
\end{table}
\end{frame}

\begin{frame}
 \frametitle<+->{Scatterplot: {\tt plot()}}
 \begin{minipage}{.48\linewidth}
 \includegraphics[width=\textwidth]{myginiplot.pdf}
 \end{minipage}\hfill
 \begin{minipage}{.48\linewidth}
 \begin{itemize}[<+->]
   \item For each year, the gini coefficient for the US
   \item Each point is an observation
   \item Each axis is a variable
   \item If you have an independent variable, it goes on the x-axis
   \item Dependent on y-axis
   \item Great `first look' at your data
 \end{itemize}
 \end{minipage}
\end{frame}

\begin{frame}
 \frametitle<+->{Scatterplot: So...}
 \begin{minipage}{.48\linewidth}
 \includegraphics[width=\textwidth]{myginiplot.pdf}
 \end{minipage}\hfill
 \begin{minipage}{.48\linewidth}
 \begin{itemize}
   \item \blue{Pattern}: \uncover<2->{curvilinear}
   \item \blue{Direction}: \uncover<3->{negative, then positive; income inequality in the US declined until the 1970s, then increased}
   \item \blue{Strength}: 
     \begin{itemize}
       \item \uncover<4->{Pretty consistent}
       \item \uncover<5->{Very few exceptions}
     \end{itemize}
 \end{itemize}
 \end{minipage}
\end{frame}

\begin{frame}
 \frametitle<+->{Scatterplot: Another one}
 \begin{minipage}{.5\linewidth}
 \includegraphics[width=\textwidth]{lifeexp.png}
 \end{minipage}\hfill
 \begin{minipage}{.48\linewidth}
 \begin{itemize}
   \item \blue{Pattern}: \uncover<2->{linear?}
   \item \blue{Direction}: \uncover<3->{positive; as income increases people generally live longer}
   \item \blue{Strength}: 
     \begin{itemize}
       \item Consistency: \uncover<4->{moderate}
       \item Exceptions: \uncover<5->{some exceptions to trend!}
     \end{itemize}
 \end{itemize}
 \end{minipage}
\end{frame}

\begin{frame}
 \frametitle<+->{Scatterplot: One more}
 \begin{minipage}{.5\linewidth}
 \includegraphics[width=\textwidth]{illiteracy.png}
 \end{minipage}\hfill
 \begin{minipage}{.48\linewidth}
 \begin{itemize}
   \item \blue{Pattern}: \uncover<2->{not obvious}
   \item \blue{Direction}: \uncover<3->{positive, maybe?}
   \item \blue{Strength}: 
     \begin{itemize}
       \item Consistency: \uncover<4->{very week}
       \item Exceptions: \uncover<5->{no obvious relationship}
     \end{itemize}
 \end{itemize}
 \end{minipage}
\end{frame}

\begin{frame}
 \frametitle<+->{Scatterplot: Be Careful}
 \begin{minipage}{.48\linewidth}
 \includegraphics[width=\textwidth]{bushreagan.png}
 \end{minipage}\hfill
 \begin{minipage}{.48\linewidth}
 \begin{itemize}
   \item Sometimes observations stack up on \blue{same} point
   \item Can \blue{hide} general pattern\\
   (looks like no relationship, but actually a moderately strong relationship!)
   \item Happens when individuals \blue{approximate} responses (round up or round down)
 \end{itemize}
 \end{minipage}
\end{frame}

\begin{frame}
 \frametitle{Making Comparisons with Interval Data}
 \begin{center}
 \includegraphics[width=\textwidth]{comp_r1.png}
 \end{center}
\end{frame}

\begin{frame}
 \frametitle{Making Comparisons with Interval Data: Features of r}
 \begin{center}
 \includegraphics[width=\textwidth]{comp_r2.png}
 \end{center}
\end{frame}

\begin{frame}
 \frametitle{Making Comparisons with Interval Data: Notice}
 \begin{center}
 \includegraphics[width=\textwidth]{comp_r2.png}
 \end{center}
\end{frame}

\begin{frame}
 \frametitle<+->{Scatterplot vs Correlation}
 \begin{minipage}{.58\linewidth}
 \includegraphics[width=\textwidth]{bushreagan.png}
 \end{minipage}\hfill
 \begin{minipage}{.38\linewidth}
 \begin{itemize}
   \item $r = 0.589$ (linear?)
   \item moderately strong, positive ($r > 0$) relationship 
 \end{itemize}
 \end{minipage}
\end{frame}

\begin{frame}
 \frametitle<+->{Scatterplot vs Correlation}
 \begin{minipage}{.58\linewidth}
 \includegraphics[width=\textwidth]{ethnicincome.png}
 \end{minipage}\hfill
 \begin{minipage}{.38\linewidth}
 \begin{itemize}
   \item $r = -0.321$ (linear?)
   \item weak, negative ($r < 0$) relationship 
 \end{itemize}
 \end{minipage}
\end{frame}

\begin{frame}
 \frametitle<+->{Scatterplot vs Correlation}
 \begin{minipage}{.58\linewidth}
 \includegraphics[width=\textwidth]{govincome.png}
 \end{minipage}\hfill
 \begin{minipage}{.38\linewidth}
 \begin{itemize}
   \item $r = 0.810$ (close to linear)
   \item strong, positive ($r > 0$) relationship 
 \end{itemize}
 \end{minipage}
\end{frame}

\begin{frame}
 \frametitle{Correlation Formula}
 \begin{center}
 \includegraphics[width=\textwidth]{corr_formula.png}
 \end{center}
 \pause
 \begin{enumerate}[<+->]
   \item Standardize every observation of X: subtract from mean of X and divide by sd of X
   \item Standardize every observation of Y: subtract from mean of Y and divide by sd of Y
   \item Multiplying (1) and (2) together
   \item Add up, then divide by sample size (N) minus one
 \end{enumerate}
\end{frame}

\begin{frame}
 \frametitle<+->{Correlation Formula Example}
 \begin{center}
 \includegraphics[width=.85\textwidth]{corr_formul_ex.png}
 \end{center}
\end{frame}

\begin{frame}
 \frametitle<+->{Correlation: R to the Rescue}
 \begin{itemize}[<+->]
   \item {\tt cor()}
   \item US inequality and political polarization (difference in mean voting patterns of democrats and republicans in Congress)
 \end{itemize}
\end{frame}

\begin{frame}
 \frametitle<+->{Correlation: Symmetry}
 \begin{itemize}[<+->]
   \item r is ``symmetric"
   \item Correlation between \alert{X} and \blue{Y}...is same as between \blue{Y} and \alert{X}
   \item Even if plots look different
   \item Correlation doesn't change according to which  variable is dependent  or independent
 \end{itemize}
 \vspace{-.2cm}
 \begin{center}
 \uncover<3->{\includegraphics[width=.85\textwidth]{lifeincome.png}}
 \end{center}
\end{frame}

\begin{frame}
 \frametitle<+->{Correlation: Unit-less}
 \begin{minipage}{.3\linewidth}
 \begin{itemize}[<+->]
   \item Correlation coefficient (r) is not expressed in terms of units of original variables
   \bigskip
   \item Makes no difference what original units of variable were
 \end{itemize}
 \end{minipage}\hfill
 \begin{minipage}{.65\linewidth}
 \uncover<4->{Example:}
 \begin{itemize}[<+->]
       \item Correlation between daily max temp in Palo Alto and Berkeley is 0.72...
       \item Can convert one city to degrees Celsius...or both to Celsius...or both to Fahrenheit (or Kelvin!)
       \item r remains 0.72!
 \end{itemize}
  \begin{center}
 \uncover<5->{\includegraphics[width=.85\textwidth]{FtoC.png}}
 \end{center}
 \end{minipage}
\end{frame}

\begin{frame}
 \frametitle<+->{Correlation Does not Imply Causation}
 \includegraphics[width=.7\textwidth]{CorrelationCausation.jpg}
 \begin{itemize}[<+->]
   \item If X and Y are correlated in the sample
   \item X may cause Y...Y may cause X...Z may cause both...pure chance (random sampling error)
 \end{itemize}
\end{frame}

\begin{frame}
 \frametitle<+->{Examining Your Data}
 \begin{itemize}[<+->]
   \item Goal: Practice working with real data in R; make the concepts we've been learning in class application to your work and life
   \item Work in groups (learn from your peers)
   \item Decided on a dataset (your own, one you've come across)
   \item Bring that dataset to class next Wed 2/24 (csv from excel or google sheets)
   \item Explore that data in class next Wed 2/24 (plots, descriptive stats)
   \item Share
 \end{itemize}
\end{frame}
%\begin{frame}
% \frametitle{Whole pic}
%% \framesubtitle{MA Senate Election (Metaxas and Mustafaraj 2010)}
% \begin{center}
% \includegraphics[width=.7\textwidth]{graphinvertu.png}
% \end{center}
%\end{frame}
%

%\begin{frame}
% \frametitle<+->{Left pic}
% \begin{minipage}{.48\linewidth}
% \includegraphics[width=\textwidth]{crying.png}
% \end{minipage}\hfill
% \begin{minipage}{.48\linewidth}
% \begin{itemize}[<+->]
%   \item TBD
%   \item TBD
% \end{itemize}
% \end{minipage}
%\end{frame}

%\begin{frame}
% \frametitle<+->{Right pic}
% \begin{minipage}{.48\linewidth}
% \begin{itemize}[<+->]
%   \item TBD
%   \item TBD
% \end{itemize}
% \end{minipage}\hfill
% \begin{minipage}{.48\linewidth}
% \includegraphics[width=\textwidth]{crying.png}
% \end{minipage}
%\end{frame}

%\begin{frame}
% \frametitle<+->{Summing Up}
% \uncover<2->{\alert{Reminders}}
% \pause
% \begin{itemize}[<+->]
%  \item Team Feedback: be specific about your observations
%  \item Lab 3: replicate the \alert{results}
% \end{itemize}
%\end{frame}

\end{document}


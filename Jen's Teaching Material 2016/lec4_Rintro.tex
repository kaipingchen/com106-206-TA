\documentclass{beamer}
\usetheme{CambridgeUS}
\usepackage{subfigure}
\usepackage{multirow}
% \usepackage{makecell}
\usepackage{graphicx}
\usepackage{xcolor}
\definecolor{liteblue}{rgb}{0.36, 0.54, 0.66}
\newcommand{\lblue}{\textcolor{liteblue}}
\usepackage{tikz}
\usepackage{CJKutf8}
\usepackage{ucs}
\usepackage[utf8]{inputenc}
% \usepackage{co}
\usepackage{multicol}
\usepackage{bm}
\usepackage{array}
\usepackage[normalem]{ulem}
\definecolor{liteblue}{rgb}{0.63,0.71,0.8}
\def\checkmark{\tikz\fill[scale=0.4](0,.35) -- (.25,0) -- (1,.7) -- (.25,.15) -- cycle;} 
\newcommand{\blue}{\textcolor{blue}}
\newcommand{\black}{\textcolor{black}}
\newcolumntype{x}[1]{%
>{\centering\arraybackslash}p{#1}}%
\def\arraybackslash{\let\\\tabularnewline}
\setbeamertemplate{itemize items}[default]
\setbeamertemplate{enumerate items}[default]
\setbeamertemplate{footline}{}
\setbeamertemplate{navigation symbols}{}%remove navigation symbols
%{\hfill\insertframenumber/\inserttotalframenumber}%
%\newcommand{\backupbegin}{
%   \newcounter{framenumberappendix}
%   \setcounter{framenumberappendix}{\value{framenumber}}
%}
%\newcommand{\backupend}{
%   \addtocounter{framenumberappendix}{-\value{framenumber}}
%   \addtocounter{framenumber}{\value{framenumberappendix}} 
%}
\makeatletter
\setbeamertemplate{footline}
{
  \leavevmode%
  \hbox{%
  \begin{beamercolorbox}[wd=.333333\paperwidth,ht=2.25ex,dp=1ex,center]{author in head/foot}%
    \usebeamerfont{author in head/foot}\insertshortauthor~~\beamer@ifempty{\insertshortinstitute}{}{(\insertshortinstitute)}
  \end{beamercolorbox}%
  \begin{beamercolorbox}[wd=.333333\paperwidth,ht=2.25ex,dp=1ex,center]{title in head/foot}%
    \usebeamerfont{title in head/foot}\insertshorttitle
  \end{beamercolorbox}%
  \begin{beamercolorbox}[wd=.333333\paperwidth,ht=2.25ex,dp=1ex,right]{date in head/foot}%
    \usebeamerfont{date in head/foot}\insertshortdate{}\hspace*{2em}
    % \insertframenumber{} / \inserttotalframenumber\hspace*{2ex} % DELETED
  \end{beamercolorbox}}%
  \vskip0pt%
}
\makeatother

\title[COMM 106/206]{Introduction to R, Day 2}
%
\subtitle{\black{Communication Research Methods}} %Comm 106/206: 
\author[Jennifer Pan]{Jennifer Pan}
%
\institute[Stanford]{Assistant Professor\\
  Department of Communication\\
  Stanford University
  \mbox{ }\\
  \mbox{ }\\
  \mbox{ }\\
  \mbox{ }\\
  \mbox{ }\\
  \mbox{ }\\
  \mbox{ }\\
  \mbox{ }\\
  \mbox{ }\\
  {January 15, 2016}}


\date{}

\begin{document}

\frame{\titlepage}
\date{15 Jan. 2016}

\begin{frame}
 \frametitle<+->{Announcements}
 \begin{itemize}[<+->]
   \item Pset 1: provide rationale for answers
   \item Friday section room (see syllabus)
 \end{itemize}
\end{frame}

\begin{frame}
 \frametitle<+->{Overview}
 \begin{itemize}[<+->]
   \item Last time
     \begin{itemize}
       \item Installed R and RStudio
       \item Started using RStudio
       \item Learned how to create / manipulate objects
     \end{itemize}
   \item Today
     \begin{itemize}
       \item Review: creating / manipulating objects
       \item Some useful functions
       \item Load datasets into R
     \end{itemize}
 \end{itemize}
\end{frame}

\begin{frame}
 \frametitle<+->{Functions}
 \begin{itemize}
   \item We have seen several functions: sqrt(), class(), c()
   \item Format: {\tt funcname(input)}
     \begin{itemize}
       \item {\tt funcname}: function name
       \item {\tt (input)}: input object (also called arguments)
     \end{itemize}
   \item Useful functions
     \begin{itemize}
       \item {\tt length()}: length of a vector or equivalently the number of its elements
       \item {\tt min()}: min value
       \item {\tt max()}: max value
       \item {\tt range()}: range of data
       \item {\tt mean()}: mean
       \item {\tt sum()}: sum all values in vector
       \item {\tt names()}: access and assign names to elements of a vector
     \end{itemize}
   \item To avoid confusion and problems stemming from the order, specify name of argument
 \end{itemize}
\end{frame}


\begin{frame}
 \frametitle<+->{Data Files}
 \begin{itemize}
   \item Vectors: manually entered data into R (not efficient)
   \item Most times: load data from an external file
   \item We will deal with two types of data
     \begin{itemize}
       \item csv: comma separated values
       \item RData: collection of R objects including datasets
     \end{itemize}
 \end{itemize}
\end{frame}

\begin{frame}
 \frametitle<+->{Local Files: Change Working Directory}
 \begin{itemize}
   \item Open RStudio
   \item Create a New R Script
   \item To load a file, you must know
     \begin{itemize}
       \item where the file is on my computer (what is the file path)
       \item copy the file path
       \item change my working directory in RStudio to the directory where the file is
     \end{itemize}
   \item Check current working directory in RStudio: {\tt getwd()}
   \item Change current working directory in RStudio: {\footnotesize{\tt setwd("username/folder/folderwithdata/")}}
 \end{itemize}
\end{frame}

\begin{frame}
 \frametitle<+->{Local Files}
 \begin{itemize}
   \item csv
     \begin{itemize}
	   \item use {\tt read.csv()}
	   \item use assignment operator to save as an object
	   \item name of object is up to you
	   \item \alert{never change the name of the data files you use in this class}
     \end{itemize}
   \item RData
     \begin{itemize}
	   \item use {\tt load()}
	   \item do not use assignment operator, R objects stored in the RData file already have object names
     \end{itemize}
 \end{itemize}
\end{frame}

\begin{frame}
 \frametitle<+->{Learn about loaded data}
 \begin{itemize}
   \item use {\tt class()}
   \item often are {\tt data.frame} objects: collection of vectors, but we can think of it like a spreadsheet
   \item useful functions
     \begin{itemize}
		\item {\tt names()}: vector of variable names
		\item {\tt nrow()}: number of rows
		\item {\tt ncol()}: number of columns
		\item {\tt dim()}: combines ncol and nrow
		\item {\tt summary()}: for each variable, the min, 25th percentile, median, 75th percentile, max
		\item {\tt View()}: same as clicking in Environment, shows data in table format
     \end{itemize}
   \item to access an individual variables of data.frame (as a vector)
     \begin{itemize}
		\item {\tt \$} operator
		\item indexing {\tt []}
     \end{itemize}
   \item missing values: {\tt NA}, some function may need {\tt na.rm$=$TRUE} to work
 \end{itemize}
\end{frame}

\begin{frame}
 \frametitle<+->{Saving Objects}
 \begin{itemize}
   \item We can write objects are .csv or RData
     \begin{itemize}
		\item For RData: {\tt save(UNpop, file = "myUNpop.RData")}
		\item For csv: {\tt write.csv(UNpop, file = "myUNpop.csv")}
     \end{itemize}
 \end{itemize}
\end{frame}

\begin{frame}
 \frametitle<+->{Packages}
 \begin{itemize}
   \item R is open source
   \item Large community of people who contribute various functionalities as \alert{R packages}
   \item Example: {\tt foreign} package to read in data from programs like STATA and SPSS
     \begin{itemize}
		\item The first time you use a package, you have to install it:\\
		{\tt install.packages(``foreign")}
	    \item \alert{Every} time you use the package (with a new session of RStudio), you have to load it:\\
	    {\tt library(``foreign")}
	    \item After package is loaded, you can use the functions to load ``foreign" data:\\
	    {\tt read.dta("UNpop.dta")}
     \end{itemize}
 \end{itemize}
\end{frame}

%\begin{frame}
% \frametitle<+->{Logical Operators}
% \begin{itemize}
%   \item TBD
%     \begin{itemize}
%		\item TBD
%     \end{itemize}
% \end{itemize}
%\end{frame}
%
%\begin{frame}
% \frametitle<+->{Relationship Operators}
% \begin{itemize}
%   \item TBD
%     \begin{itemize}
%		\item TBD
%     \end{itemize}
% \end{itemize}
%\end{frame}
%
%\begin{frame}
% \frametitle<+->{Subsetting}
% \begin{itemize}
%   \item TBD
%     \begin{itemize}
%		\item TBD
%     \end{itemize}
% \end{itemize}
%\end{frame}
%
%\begin{frame}
% \frametitle<+->{Simple conditional statements}
% \begin{itemize}
%   \item TBD
%     \begin{itemize}
%		\item TBD
%     \end{itemize}
% \end{itemize}
%\end{frame}
%
%\begin{frame}
% \frametitle<+->{Factor variables}
% \begin{itemize}
%   \item TBD
%     \begin{itemize}
%		\item TBD
%     \end{itemize}
% \end{itemize}
%\end{frame}


%\begin{frame}
% \frametitle{Whole pic}
%% \framesubtitle{MA Senate Election (Metaxas and Mustafaraj 2010)}
% \includegraphics[width=.8\textwidth]{crying-baby.jpg}
%\end{frame}
%
%\begin{frame}
% \frametitle<+->{Left pic}
% \begin{minipage}{.48\linewidth}
% \includegraphics[width=\textwidth]{crying-baby.jpg}
% \end{minipage}\hfill
% \begin{minipage}{.48\linewidth}
% \begin{itemize}[<+->]
%   \item TBD
%   \item TBD
% \end{itemize}
% \end{minipage}
%\end{frame}
%
%\begin{frame}
% \frametitle<+->{Right pic}
% \begin{minipage}{.48\linewidth}
% \begin{itemize}[<+->]
%   \item TBD
%   \item TBD
% \end{itemize}
% \end{minipage}\hfill
% \begin{minipage}{.48\linewidth}
% \includegraphics[width=\textwidth]{crying-baby.jpg}
% \end{minipage}
%\end{frame}

%\begin{frame}
% \frametitle<+->{Summing Up}
% \uncover<2->{\alert{Reminders}}
% \pause
% \begin{itemize}[<+->]
%  \item Team Feedback: be specific about your observations
%  \item Lab 3: replicate the \alert{results}
% \end{itemize}
%\end{frame}

\end{document}

